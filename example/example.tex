\documentclass[compress,t]{beamer}

\usetheme{usyd}
%\usetheme[nosection]{usyd} % use this one for no automatic section slides

\title{Presentation title}
\subtitle{Presentation subtitle}
\author{Firstname Lastname}
\institute{Faculty, centre or unit}

\begin{document}

  % Make title frame
  \setbeamercolor{background canvas}{bg=usydorange}
  \begin{frame}[plain,b]
    \titlepage
  \end{frame}
  \setbeamercolor{background canvas}{bg=white}

  \section{Introduction}
  \subsection{Subsection 1}
  \begin{frame}{Section 1 - Subsection 1 - Frame 1}
    \framesubtitle{Frame Subtitle}
    \begin{itemize}
      \item Item 1
      \item Item 2
      \item Item 3
    \end{itemize}
  \end{frame}
  \begin{frame}{Section 1 - Subsection 1 - Frame 2}
    \begin{itemize}
      \item Item 1
      \item Item 2
      \item Item 3
    \end{itemize}
  \end{frame}
  \subsection{Subsection 2}
  \begin{frame}{Section 1 - Subsection 2 - Frame 1}
    \begin{itemize}
      \item Item 1
      \item Item 2
      \item Item 3
    \end{itemize}
  \end{frame}

  \section{Section 2}
  \subsection{Subsection 1}
  \begin{frame}{Section 2 - Subsection 1 - Frame 1}
    $$\Gamma(t) = \int_0^\infty x^{t-1} e^{-x}\, \mathrm{d} x$$
    $$\int_0^1 \ln \Gamma(t)\, \mathrm{d} t = \frac{1}{2}\ln 2\pi$$
  \end{frame}
  \subsection{Subsection 2}
  \begin{frame}{Section 2 - Subsection 2 - Frame 1}
    \begin{theorem}
      Let $G = (V,E)$ be a graph and $\deg(u)$ denote the degree of a vertex $u \in V$, then $$\sum_{u \in V} \deg(u) = 2 \vert E \vert.$$
    \end{theorem}
  \end{frame}

\end{document}
